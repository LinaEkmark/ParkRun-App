% Vincent skriver här

% Problem/Uppgift
% Det här avsnittet är ofta den viktigaste delen av planeringsrapporten (och av den slutgiltiga uppsatsen/rapporten). Den syftar till att identifiera frågan/frågorna som ska tas upp i projektet. Det är viktigt att gruppen gör en problemanalys även om det i projektförslaget redan finns ett problem (en uppgift) specificerat. Anledningen till detta är att det riktiga primära problemet ofta skiljer sig från det i början av uppdragsgivaren/förslagsställaren/kunden föreslagna. Problemanalysen syftar också till att bryta ner problemet/uppgiften i mindre och mer detaljerade delproblem/deluppgifter, vilket också leder till formulering av delsyften. Genom att göra detta får studenterna mycket bättre förståelse för de olika aspekterna av problemet/uppgiften. Utan den här informationen är det omöjligt att identifiera vilken information som behövs, vilka informationskällor som behövs och lämpliga tillvägagångssätt.
%En bra problemanalys som identifierar delproblem/deluppgifter och delsyften vilar i många fall på användning av teorier och modeller från litteraturen. En litteraturgenomgång bör därför genomföras tidigt i processen.

\section{Problem}
Det huvudsakliga problem som kunden vill åtgärda är svårigheterna kring att placera ut markörer på rätt platser längs banan. Detta kan brytas ner i flera delproblem:
\begin{itemize}
    \item \textbf{Delproblem 1: Hitta till rätt plats.}\\
        Att lösa detta problem kräver någon form av navigationsfunktion i appen. Det är även fördelaktigt att appen visar den enklaste vägen mellan markörer (vilket inte nödvändigtvis är samma väg som banan).  
    \item \textbf{Delproblem 2: Sätta ut markören på rätt sätt.}\\
        Även om volontären hittar till rätt plats kan det vara svårt att placera markören på ett sätt som fungerar. Därför är det viktigt att appen även kan ge tydliga instruktioner för hur markören placeras ut korrekt på varje enskild plats. 
    \item \textbf{Delproblem 3: Meddela att markören är uppsatt.}\\
        Koordination inom volontärlaget är viktigt för att arbetet skall kunna flyta på. Därav kan det vara underlättande för volontärerna om de kan meddela sin markör som uppsatt via appen.
\end{itemize}
Nästa problem ligger i att detta verktyg skall fungera för fler än ett parkrun, det skall alltså finnas möjlighet för event-organisatörer för olika parkruns att lägga in sina banor i appen. Även detta kan brytas ner i flera delproblem:
\begin{itemize}
    \item \textbf{Olika typer av användare}\\
        Vem som helst får inte skapa ett eget parkrun. Det finns en utarbetad, tydlig process som följs varje gång ett nytt parkrun skapas. Detta måste efterlevas även i appen - bara officiella parkruns skall kunna skapas. 
\end{itemize}