% Vincent skriver här

% Problem/Uppgift
% Det här avsnittet är ofta den viktigaste delen av planeringsrapporten (och av den slutgiltiga uppsatsen/rapporten). Den syftar till att identifiera frågan/frågorna som ska tas upp i projektet. Det är viktigt att gruppen gör en problemanalys även om det i projektförslaget redan finns ett problem (en uppgift) specificerat. Anledningen till detta är att det riktiga primära problemet ofta skiljer sig från det i början av uppdragsgivaren/förslagsställaren/kunden föreslagna. Problemanalysen syftar också till att bryta ner problemet/uppgiften i mindre och mer detaljerade delproblem/deluppgifter, vilket också leder till formulering av delsyften. Genom att göra detta får studenterna mycket bättre förståelse för de olika aspekterna av problemet/uppgiften. Utan den här informationen är det omöjligt att identifiera vilken information som behövs, vilka informationskällor som behövs och lämpliga tillvägagångssätt.
%En bra problemanalys som identifierar delproblem/deluppgifter och delsyften vilar i många fall på användning av teorier och modeller från litteraturen. En litteraturgenomgång bör därför genomföras tidigt i processen.

\section{Problem}
Det huvudsakliga problem som kunden vill åtgärda är att förenkla arbetet för volontärerna som förbereder och arrangerar varje lopp. I nuläget är den främsta svårigheten kring att placera ut markörer på rätt platser längs banan. Detta kan brytas ner i flera delproblem:
\begin{itemize}
    \item \textbf{Delproblem 1: Hitta till rätt plats.}\\
        Eftersom att Parkruns vanligtvis arrangeras i parker eller skogar är det ofta svårt att hitta till rätt plats. Ofta kan det saknas foton från Google-maps och ibland kan platsen även ha en svag GPS-täckning således kan det vara svårt att hitta till rätt plats. 
        Att lösa detta problem kräver någon form av navigationsfunktion i appen. Det är även fördelaktigt att appen visar den enklaste vägen mellan markörer (vilket inte nödvändigtvis är samma väg som banan).  
    \item \textbf{Delproblem 2: Sätta ut markören på rätt sätt.}\\
        Även om volontären hittar till rätt plats kan det vara svårt att placera markören på ett sätt som fungerar. Då det är av högsta prioritet att rätt markör hamnar på rätt ställe för att annars springer löparna fel. Därför är det viktigt att appen även kan ge tydliga instruktioner för hur och vara väldigt specifik i hur markören placeras ut korrekt på varje enskild plats. 

        %Add a sentence or two to explain why it is hard to find the right place. The majority of parkruns take place in parks or scenic areas and often GPS coverage is not good, there are no google map images, and there are no clear visual landmarks to help accurately pinpoint the location. (I can provide a few pictures of Skatås route if that helps to demonstrate the point.)
    \item \textbf{Delproblem 3: Meddela att markören är uppsatt.}\\
        Koordination inom volontärlaget är viktigt för att arbetet skall kunna flyta på. Därav kan det vara underlättande för volontärerna om de kan meddela sin markör som uppsatt via appen.

    \item \textbf{Del problem 4: Appens användarvänlighet} \\
    För att få att volontärerna använda  appen krävs det att den är intuitiv och användarvänlig och attraktiv. Detta adresseras genom att intervjuer har utfärdas i ett tidigt stadie med volontärer. Intervjuer för att få in deras synpunkter kring layout och eventuella förbättringsmöjligheter. Fler intervjuer är planerade med fler volontärer och organisatörer från flera olika Parkruns. Detta för att kontinuerligt under utvecklingsprocessen kunna möta kundernas önskan.


\end{itemize}
Nästa problem ligger i att detta verktyg skall fungera för fler än ett parkrun, det skall alltså finnas möjlighet för event-organisatörer för olika parkruns att lägga in sina banor i appen. Även detta kan brytas ner i flera delproblem:
\begin{itemize}
    \item \textbf{Olika typer av användare}\\
        Vem som helst får inte skapa ett eget parkrun. Det finns en utarbetad, tydlig process som följs varje gång ett nytt parkrun skapas. Detta måste efterlevas även i appen - bara officiella parkruns skall kunna skapas. 
\end{itemize}


        % Nånting nånting skapa kartor
        % - från google my maps?
        % - från grunden?
        % nånting enkel process för tekniskt obegåvade
