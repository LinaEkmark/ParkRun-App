\section{Metod}




\subsection{Mål}
Målet med detta kandidatarbete är att leda utvecklingen av en mobilapplikation från start till färdig produkt. Processen omfattar regelbundna veckomöten internt inom kandidatarbetsgruppen samt möten med handledaren för att säkerställa framsteg och tydlighet genom hela arbetsprocessen. Målgruppen för den planerade applikationen är volontärer och organisatörer av Parkrun. Denna användargrupp spänner över en bred ålderskategori, och därför är vårt mål att skapa en applikation som är användarvänlig oavsett användarens erfarenhetsnivå med mobilapplikationer.



\subsection{Genomförande}

Projektet kommer genomföras med hjälp av intervjuer och egen kreativitet utifrån Parkruns hemsida. 
Under planeringsfasen av arbetet så kommer intervjuer med volontärer anordnas för att få den angivna målgruppens åsikter och önskemål. 
Dessa åsikterna kommer ligga till grupp för majoriteten av appens funktionalitet för att skapa en så användbar applikation som möjligt. 

En första prototyp av produkten kommer skapas i Figma som är ett designverktyg. Denna modellen kommer sedan visas under intervjuerna för 
kunna få en direkt feedback på denna som sedan kan användas som grund under resterande designprocess.

Vid behov kommer fälttester utföras för att säkerställa att målet med en användarvänlig produkt uppnås. Detta kommer genomföras genom att 
efterfråga att Parkruns volontärer tester applikationen för att ge våran målgrupp en praktisk erfarenhet som sedan kan generera konstruktiv feedback. 
Med dessa studierna till grund kan applikationens positiva utveckling säkerställas. 

\textbf{\textit{Här tänker jag att vi fortsätter skriva om hur vi tänker gå till väga med arbetet.}}

\textit{Mina tankar är:
\begin{itemize}
    \item Hur veckorna kommer se ut (sprinter)
    \item Vilka tekniker vi kommer använda (programspråk? Bibliotek? Design? etc)
    \item Hur delar vi upp arbetet i gruppen?
    \item Koppla samman tidsplan med genomförande?
\end{itemize}}

\subsection{Datainsamling}

\textbf{\textit{Här borde vi kolla om vi kan få tillgång till någon form av databas från parkrun för kartor och kanske om volontärer som anmält sig och sånt}}

% SMART-mål (Specifika, Mätbara, Accepterade, Realistiska och Tidsbegränsade)
\subsection{User Stories}

Som en användare önskar jag att tydligt kunna se vilken roll jag har, så att jag är medveten om mina arbetsuppgifter och vet vad jag kan åstadkomma i appen. Detta syftar till att skapa en enklare förståelse för appens olika roller och underlättar för mig att hålla koll på mina specifika uppgifter.

\vspace{1em}
Som en användare önskar jag att det inte tar mer än två klick att nå viktig information, såsom kartan eller mitt ansvarsområde på banan. Målet är att göra appen simpel och användarvänlig. Detta är särskilt viktigt för mig som inte är van vid att använda appar, så att jag lätt kan hitta information utan att bli förvirrad.

\vspace{1em}
Som en administrator önskar jag att kunna hantera användare i appen genom att kunna ta bort anmälningar av olika sträckor på kartan. Målet är att ha fullständig kontroll över funktionerna för att snabbt och effektivt lösa eventuella tekniska problem som kan uppstå.

\vspace{1em}
Som en volontär önskar jag att tydligt kunna se min tilldelade del av banan genom att den markeras med en annan färg på kartan. Detta gör det enkelt för mig att visuellt uppfatta och identifiera området där jag har ansvar under evenemanget.

\vspace{1em}

Som volontär vill jag kunna markera en "point-of-interest" längs banan på kartan med högst 3 klick. Genom att göra detta i appen ska jag enkelt kunna kommunicera varningar eller upplysningar, som till exempel ett fallet träd. Vid markering på kartan ska jag kunna välja typ av "point-of-interest" och skriva in ett meddelande för att tydligt förmedla min upplysning.
\vspace{1em}

Som administratör vill jag se tydiga markeringar på kartan som indikerar vilken typ av "point-of-interest" som har markerats. För att uppnå detta önskar jag att olika färger och symboler används på markörerna. Genom dessa visuella indikatorer blir det enkelt för mig att snabbt förstå vad som har inträffat och vilka åtgärder som eventuellt behöver vidtas.



%Som en ...
%vill jag att ...
%så att ...
%när jag ...
%händer detta ...