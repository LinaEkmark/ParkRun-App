\section{Metod}

%Hur gruppen har tänkt sig att genomföra arbetet är val av metod. I konstruktionsinriktade projekt kan detta tyckas vara självklart, men det kan även i detta fall finnas viktiga metodval. Helt litteraturbaserade kandidatarbeten är också genomförbara men även en litteraturstudie ska ha en
%ordnad och strukturerad arbetsprocess och metodik.

%Metodavsnittet bör också beskriva hur data ska samlas in och hur det konstateras hur väl projektets mål har uppfyllts. I praktiska projekt kan detta vara genom mätningar av olika typer. Det kan också vara genom datorsimuleringar. Vilka aspekter är viktiga för att veta om syftet med projektet har uppnåtts? Datainsamling kan också vara en del av en testning eller annan utvärdering av den produkt som tas fram i ett konstruktionsinriktat projekt.

%Antal studieobjekt/testfall och hur de väljs? Typ av undersökningsmetod/testmetod? Hur insamlade data/testresultat ska analyseras och presenteras? Hur ser processen ut för litteraturarbetet?


I ett projekt med tidspress är det avgörande att bryta ner en stor uppgift i mindre delar för att underlätta utvecklingen \cite{agile}. För att säkerställa efterlevnaden av denna strategi för problemlösning, måste projektet välja en metod som effektivt implementerar detta tillvägagångssätt. Metoden som har valts är agil systemutveckling. Detta arbetssättet använder sig av att bryta ner stora problem i mindre utmaningar, och genom att använda sig av sprinter kan man säkerställa att arbetet hela tiden rör dig framåt. Enligt OpenText medför denna metoden en bristfällig dokumentation vilket är något som inte är lämplig för denna typen av projekt \cite{agile}.  Av denna anledningen kommer inte projektet följa en strikt agil utvecklingsmetod, utan kommer istället enbart ta inspiration av dess karakteristiska sprinter som primär veckouppbyggnad.

\subsection{Genomförande}

Projektet kommer genomföras med hjälp av intervjuer och egen kreativitet utifrån Parkruns hemsida. Under planeringsfasen av arbetet kommer intervjuer med volontärer genomföras för att tidigt få feedback och insikter om vilka funktionen målgruppen är intresserad av. En första prototyp av produkten kommer skapas i Figma som är ett designverktyg. Denna modellen kommer sedan visas under intervjuerna för kunna få en direkt feedback på denna som sedan kan användas som grund under resterande designprocess. Fälttester kommer även utföras för att säkerställa att målet med en användarvänlig produkt uppnås. Detta kommer genomföras genom att efterfråga att Parkruns volontärer testar applikationen för att ge vår målgrupp en praktisk erfarenhet som sedan kan generera konstruktiv feedback. Med dessa studierna till grund kan applikationens positiva utveckling säkerställas\\

Alla veckor kommer utgå ifrån samma struktur. Måndagar kommer bestå av  ett planeringsmöte där veckans arbetsuppgifter kommer delas ut till alla gruppmedlemmar. Dessa kommer att tilldelas baserat på intresse, erfarenhet och ansvarsområden inom gruppen, med syfte att skapa ett engagerande och effektivt arbetsklimat för alla gruppmedlemmar. På onsdagar kommer ett veckomöte hållas med projektets handledare för att diskutera eventuella frågor och tankar om arbetet. Dessa frågor kan då besvaras och användas för att fortsätta utvecklingsprocessen. Veckan kommer sedan avslutas med ett standup möte på fredagar där varje gruppmedlem redovisar för gruppen vad som hen har åstadkommit under veckan. Under detta tillfälle kommer det även finnas tid för eventuella frågor som bör diskuteras i gruppen. Tisdagar och Torsdagar lämnas öppna för eget arbete för att prioritera tid till utveckling av produkten.\\

Detta projektet kommer skrivas med React Native för att kunna stödja både Android och IOS-plattformar utan att behöva två olika programkoder \cite{react}. React Native är ett välkänt programspråk för mobilapplikationsutveckling, vilket innebär att det finns omfattande stöd tillgängligt på internet för att svara på eventuella frågor. Dess grund bygger starkt på JavaScript, ett språk som är enkelt att både läsa och lära sig, vilket minimerar inlärningskravet. Detta är avgörande för att optimera utvecklingstiden för appen och minimera behovet av omfattande inlärning. React Native har även ett packet för kartvy som heter React Native Mapview. Detta är ett kostnadsfritt packet som integrerar en kartvy direkt i appen, vilket är avgörande då applikationen har en viktigt funktion i att visa information på en karta. Projektet kommer använda sig av programmet EXPO för att agera server för appen under dess utveckling. EXPO är ett program som körs direkt i terminalen på datorn och kopplas sedan samman med valfri telefon genom att skanna en QR-kod i EXPO GO's app. Programmet kan sedan köras direkt i telefonen behåller sedan sin anslutning så länge som servern körs. Detta är avgörande för att enkelt kunna inspektera appens utseende vid varje utvecklingssteg, oavsett vilket operativsystem som används. För att skriva snygga rapporten kommer Latex användas för rapportskrivning. Detta då det är gratis genom Chalmers och familjärt för alla gruppmedlemmar.

\subsection{Datainsamling}

Under loppet av projektet kommer flertal intervjuer och demos hållas med volontärer

% SMART-mål (Specifika, Mätbara, Accepterade, Realistiska och Tidsbegränsade)
\subsection{User Stories}


Som en användare önskar jag att tydligt kunna se vilken roll jag har, så att jag är medveten om mina arbetsuppgifter och vet vad jag kan åstadkomma i appen. Detta syftar till att skapa en enklare förståelse för appens olika roller och underlättar för mig att hålla koll på mina specifika uppgifter.

\vspace{1em}
Som en användare önskar jag att det inte tar mer än två klick att nå viktig information, såsom kartan eller mitt ansvarsområde på banan. Målet är att göra appen simpel och användarvänlig. Detta är särskilt viktigt för mig som inte är van vid att använda appar, så att jag lätt kan hitta information utan att bli förvirrad.

\vspace{1em}
Som en administratör önskar jag att kunna hantera användare i appen genom att kunna ta bort anmälningar av olika sträckor på kartan. Målet är att ha fullständig kontroll över funktionerna för att snabbt och effektivt lösa eventuella tekniska problem som kan uppstå.

\vspace{1em}
Som en volontär önskar jag att tydligt kunna se min tilldelade del av banan genom att den markeras med en annan färg på kartan. Detta gör det enkelt för mig att visuellt uppfatta och identifiera området där jag har ansvar under evenemanget.

\vspace{1em}

Som volontär vill jag kunna markera en "point-of-interest" längs banan på kartan med högst 3 klick. Genom att göra detta i appen ska jag enkelt kunna kommunicera varningar eller upplysningar, som till exempel ett fallet träd. Vid markering på kartan ska jag kunna välja typ av "point-of-interest" och skriva in ett meddelande för att tydligt förmedla min upplysning.
\vspace{1em}

Som administratör vill jag se tydiga markeringar på kartan som indikerar vilken typ av "point-of-interest" som har markerats. För att uppnå detta önskar jag att olika färger och symboler används på markörerna. Genom dessa visuella indikatorer blir det enkelt för mig att snabbt förstå vad som har inträffat och vilka åtgärder som eventuellt behöver vidtas.

\vspace{1em}
Som eventansvarig utan särskild teknisk kunskap vill jag kunna föra över kartan över mitt lokala lopp till appen, så att vi volontärer kan börja använda den. 


\subsection{Mål}
Målet med detta kandidatarbete är att leda utvecklingen av en mobilapplikation från start till färdig produkt. Processen omfattar regelbundna veckomöten internt inom kandidatarbetsgruppen samt möten med handledaren för att säkerställa framsteg och tydlighet genom hela arbetsprocessen. Målgruppen för den planerade applikationen är volontärer och organisatörer av Parkrun. Enligt vårt intervjuunderlag är majoriteten av volontärer vid Parkrun Skatås är 30 eller äldre, men bland banfunktioner är snittåldern istället 60 och över. Detta mönster gäller även Billdals Parkrun där nästan alla ledare är runt 60 års åldern. Med en äldre målgrupp i åtanke är det därför viktigt att inte göra våran app komplicerad. Vårt övergripande mål är därför att skapa en applikation som är användarvänlig oavsett användarens erfarenhetsnivå med mobilapplikationer.


%Som en ...
%vill jag att ...
%så att ...
%när jag ...
%händer detta ...