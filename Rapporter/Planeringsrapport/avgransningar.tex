\section{Avgränsningar}
I uppdraget ingår ett valfritt önskemål om att kunna se markörernas avsedda placering, med hjälp av anvisningar som dyker upp ovanpå en bild som tas av telefonkameran, medelst GPS-positionering. 
Denna typ av förstärkt verklighet kan underlätta för volontären som ska hitta rätt placering, men dagens GPS-teknik är begränsad i hur exakt positionering som går att uppnå, i synnerhet i skogsmiljö där många Parkrun-lopp går.

I kombination med den tekniska utmaningen att implementera en förstärkt verklighet-funktion av denna typ, har detta mål valts bort för att fokus ska kunna läggas på kartanvisningar och statiska bilder, som tillsammans är mer pålitliga. 
Förstärkt verklighet-funktionen kan dock komma att implementeras i mån av tid senare under projektets gång.

Utöver de beställda funktionerna dök flera idéer upp under planeringen, med potential att underlätta %osäker om ett för eller med ska hit
volontäruppdraget.
Den första idén är en möjlighet för volontärer att placera ut markörer på kartan för att förmedla något av intresse. 
Vad detta skulle kunna handla om är var det skulle vara lämpligt med skyltning i framtida lopp, om det är träd i vägen eller var det ofta sker olyckor, eventuellt med någon möjlighet att föra statistik över tillbud.
Efter dialog med Parkrun-volontär ströks den sistnämnda idén då olyckor visat sig vara extremt ovanliga.
Systemet med återkoppling via kartan kommer dock jobbas vidare på.

En möjlighet för volontärer att anmäla sig till ett lopp genom en knapp i appen var också påtänkt. 
De olika uppgifterna skulle då kunna tilldelas direkt med hjälp av appen. 
Emellertid finns ett system för anmälningar och uppgiftsfördelning genom epost och tabeller, som bedömdes omständig att integrera med appens tilltänkta anmälningsfunktion.
Idén omarbetades till att vara en genväg till anmälan via epost.
