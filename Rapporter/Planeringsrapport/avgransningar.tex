\section{Avgränsningar}
I uppdraget ingår ett valfritt önskemål om att kunna se markörernas avsedda placering, med hjälp av anvisningar som dyker upp genom augmented reality (AR). 

Anvisningarna hade då hittat sin plats med GPS-postionering.
Denna typ av AR kan underlätta för volontären som ska hitta rätt placering, men dagens GPS-teknik är begränsad i hur exakt positionering som går att uppnå, i synnerhet i skogsmiljö där många Parkrun-lopp går.

I kombination med den tekniska utmaningen att implementera en AR-funktion av denna typ, har detta mål valts bort. Istället kommer fokus ligga på att visa kartanvisningar och bilder, som tillsammans är mer pålitliga. Förstärkt verklighet-funktionen kan dock komma att implementeras i mån av tid senare under projektets gång.

Utöver de funktioner som formulerades i tesen dök flera nya idéer upp under planeringen, med potential att underlätta volontäruppdraget. Den första idén är en möjlighet för volontärer att placera ut markörer på kartan för att förmedla information. Vad detta kan handla om är var det skulle vara lämpligt med skyltning i framtida lopp, om det är träd i vägen eller var det ofta sker olyckor, eventuellt med någon möjlighet att föra statistik över incidenter. Efter dialog med Parkrun-volontär ströks den sistnämnda idén då olyckor visat sig vara extremt ovanliga. Möjligheten för ett systemet med återkoppling via kartan kräver vidare undersökning.

En möjlighet för volontärer att anmäla sig till ett lopp genom en knapp i appen var också planerad. De olika uppgifterna skulle då kunna tilldelas direkt med hjälp av appen. Emellertid finns det ett system för anmälningar och uppgiftsfördelning genom e-post och tabeller, som bedömdes omständig att integrera med appens tilltänkta anmälningsfunktion. Idén omarbetades till att vara en genväg till anmälan via e-post.
