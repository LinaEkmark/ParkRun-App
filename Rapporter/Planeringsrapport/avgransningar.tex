\section{Avgränsningar}
I uppdraget ingår ett valfritt önskemål om att kunna se markörernas avsedda placering med hjälp av anvisningar som dyker upp ovanpå en bild som tas av telefonkameran, med hjälp av GPS-positionering. 
Denna typ av förstärkt verklighet kan underlätta för volontären som ska hitta rätt placering, men dagens GPS-teknik är begränsad i hur exakt positionering som går att uppnå, i synnerhet i skogsmiljö där många Parkrun-lopp går.

I kombination med den tekniska utmaningen att implementera en förstärkt verklighet-funktion av denna typ har detta mål valts bort för att fokus ska kunna läggas på kartanvisningar och statiska bilder, som tillsammans är mer pålitliga. 
Förstärkt verklighet-funktionen kan dock komma att implementeras i mån av tid senare under projektets gång.

Utöver de beställda funktionerna dök flera idéer upp under planeringen, med potential att underlätta %osäker om ett för eller med ska hit
volontäruppdraget.

