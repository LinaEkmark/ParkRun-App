% Vincent skriver här
\section{Bakgrund}


%When adding to this, we need to discuss existing similar apps, challenges for the project, and discuss findings from the literature which will be taken into account/ used for the benefit of the project.

%Things I think would be useful to discuss would be the issues of setting up a route that is in a nature setting and possibly a forested area where GPS is poor and it is hard to identify a place accurately from photographs. The references from the previous two parkrun projects would be useful for this.

%The second thing is the users of the app and making the user experience as simple as possible for them. There you can cite books and research papers on UX / UI design of apps, both general principles and more specific to this app for example the Admin side of things letting the user upload and change things on a map and customise the route sections etc - has there been any research on this? Is there any research about using apps outside in scenic areas? Any suggestions for best practice in these situations? 

% Bakgrund ska innehålla en motivering till varför det valda ämnet är intressant 
% ur akademisk synvinkel och/eller ur tekniskt perspektiv eller i förekommande 
% fall ur kundens/uppdragsgivarens perspektiv. I vissa fall ska den här rubriken 
% inkludera en kort historik över ämnet. Efter att ha läst bakgrunden ska alla 
% läsare förstå varför ämnet är relevant. Följande frågeställningar bör vara 
% aktuella: Vad är ämnet/problemet som ska undersökas? Varför har 
% ämnet/problemet uppkommit? Varför och för vem är det ett relevant eller 
% intressant ämne/problem? Kan det specifika ämnet/problemet relateras till en 
% mer generell diskussion?

% Fokus på kunden/uppdragsgivarens perspektiv
Loppet grundades i Bushy Park, London år 2004 och spreds därefter världen över. Deltagarna träffas varje lördagsmorgon för att springa 5km på en utmarkerad sträcka. Varje enskilt lopp drivs av volontärer som fyller de olika nödvändiga uppgifter som krävs under evenemanget - de sätter ut markörer längs banan, skannar deltagares sträckkoder, ser till att alla är säkra och kommer i mål. Volontärerna kan även stå uppställda på utvalda platser längs banan för att öka säkerheten samt se till att deltagare inte springer vilse. Parkrun har 9 miljoner registrerade medlemmar med över 500.000 volontärer \href{https://www.parkrun.com/about/}{(enligt Parkrun själva)}. \\
En återkommande svårighet för volontärerna finnes i utmarkeringen av banan; det är svårt att koordinera på ett bra sätt att alla markörer faktiskt sätts ut på rätt platser. Därtill kan det vara klurigt för volontärerna att veta var de ska stå längs banan under loppet, samt att koordinera med de andra volontärerna. Det finns därav behov av verktyg som kan underlätta samt effektivisera arbetet.

% Formulering jag skrev i Bakgrund som kanske passar bättre i detta avsnitt - Vincent
% --------------------------------------------------------------------
% Detta projekt har för avsikt att utforma och skapa en mobil-app som kan
% underlätta arbetet för dessa volontärer, med störst fokus på att sätta ut 
% markörerna på rätt platser, stå på rätt ställen längs banan men även att 
% effektivisera koordination av arbetslaget. 
% --------------------------------------------------------------------