\documentclass{article}
\usepackage[utf8]{inputenc}
\usepackage{enumitem}

\title{\textbf{Intervju med volontär}}
\author{Parkrun App}
\date{}

\begin{document}

\maketitle
\thispagestyle{empty}
\section*{Information}
\begin{itemize}[label=]
    \item \textbf{Namn:}  Anna-Carin och Janne 
    \item \textbf{Ålder:} 57 61
\end{itemize}

\section*{Intervjufrågor}
\begin{enumerate}[label=\textbf{Fråga \arabic*:}]
    \item Hur länge har du deltagit som löpare i Parkrun?\\
        Sex och ett halft år(sedan skatås starade) men har sprungit till och från hela tiden. Anna-Carin är ingen större löpare utan mer av en volontär. 
    \item Hur länge har du varit volontär?\\
        \textit{svar här} Anna-Carin har varit volontär i 4-5 år. Janne organiserar i Blldal tillsammans med Anna-Carin och han har varit volontär 150 ggr sprungit 50-60
    \item Vad motiverar dig mest i volontärarbetet?\\
        \textit{svar här} Alla blir väldigt glada av det och kul grej med fika efteråt. 
Första 6:e Juni 62 pers 220 pers förra året. Att vara volontär sprider glädje och man blir del av en gemenskap. Många turister är också volontärer. Går runt på stora dagar.
    \item Kan du dela med dig av någon svårighet du stött på under ditt arbete som volontär?\\
        \textit{svar här} Inga större svårigheter, skotta snö för att förbereda. Janne saltar och skottar, för att de som reser inte ska behöva ha rest till ett inställt lopp
    \item Finns det något återkommande moment i arbetet som du känner kan effektiviseras?\\
        \textit{svar här} De har en karta med pilar och skyltar 7-8 personer som kan göra det man får visa nya. Det är en ritad karta inplastad.  En erfaren går med första tvågångarna 
    \item Känner du att en mobil-app specifikt framtagen för volontärer hade kunnat göra arbetet lättare?\\
    \textit{Om nej, gå vidare till överiga kommentarer}\\
        \textit{svar här} Tror det kan förenkla för nybörjare i nuläget finns det en kärngrupp. Parkruns egna app används redan. 
    \item Finns det några specifika funktioner du anser att en sådan app borde ha?\\
        \textit{svar här} I parkruns app finns tidtagning. Var man startar hur långt det är till busshållsplats vanlig fråga från tillresta. Göra den enkel och lätthanterad.  
    \item Vad har du för mobiltelefon? Modell, märke, etc.\\
        \textit{svar här}  Android båda två har av märket Motorola 
\end{enumerate}

\newpage
\section*{Designfrågor - visa prototypen}
\begin{enumerate}[label=\textbf{Fråga \arabic*:}]
    \item Vad tycker du om denna inloggningsskärm?\\
        \textit{Vi tycker det ser bra ut. Väldigt bra att det både finns på svenska och engelska}
    \item Vad tycker du om denna skärm för att skapa konto?\\
        \textit{Glömde fråga}
    \item Vad tycker du om vår Running events sida?\\
        \textit{Det ser okej ut. Gillar att det tydligt visas när ett event är cancelled. Vi tycker att det skulle finnas teman vid högtider.}
    \item Vad tycker du om detta som Event map sida?\\
        \textit{Vi tycker att den ser ut att fylla sina funktioner. Vi skulle gärna vilja att man kunde se en bild på hur pilen ska vara uppsatt vid varje markör. Det är inte så enkelt för första gångs volontärer att veta hur en pil ska sitta vid en stolpe så det skulle vara bra med en exempel bild}
    \item Vad tycker du om färger och typsnitt?\\
        \textit{Det liknar Parkruns hemsida vilket är trevligt och familijärt.}
\end{enumerate}

\section*{Övriga kommentarer och reflektioner}
\begin{itemize}[label=]
    \item Finns det något annat du vill lägga till eller diskutera som inte täcktes av de tidigare frågorna?\\
    \textit{Vi vill gärna att det ska finnas en markör för vårt kafe. De är välkänt över hela europa att vi har ett bra. Alla som inte kommer från sverige brukar vara väldigt imponerade av att allt är hembakat}
\end{itemize}
\thispagestyle{empty}

\end{document}