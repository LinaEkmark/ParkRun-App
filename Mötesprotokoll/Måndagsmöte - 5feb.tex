\documentclass[a4paper, 11pt]{article}
\usepackage{comment} % enables the use of multi-line comments (\ifx \fi) 
\usepackage{fullpage} % changes the margin
\usepackage[swedish]{babel}
\usepackage[utf8]{inputenc}
\usepackage{graphicx}
\usepackage{multicol}
\usepackage{float}
\usepackage{fancyhdr}
\usepackage{enumitem}
\pagestyle{fancy} 
\usepackage{pdfpages}
%\usepackage[head=128pt]{geometry}
\title{Mötesprotokoll}
\author{Axel Carlstedt}
\usepackage{geometry}
\setlength{\footskip}{0.1pt}
\setlength{\headheight}{80pt}
\setlength{\topmargin}{0pt}
\setlength\parindent{0pt}
\fancypagestyle{style1}{

\rhead{Mötesprotokoll\\}
\renewcommand{\headrulewidth}{0pt}
\cfoot{
\makebox{}\\
\makebox{}\\
\hspace{1cm}\\}
}
\fancypagestyle{style2}{

\rhead{Mötesprotokoll\\}
\renewcommand{\headrulewidth}{0pt}
\cfoot{}
}
\begin{document}
\pagestyle{style1}


\textbf{Datum:} 20xx-xx-xx\\
\textbf{Tid:} xx.xx-xx.xx\\
\textbf{Plats:} Sal / Distans

\makebox[\linewidth]{\rule{\linewidth}{0.4pt}}\\

\section*{§1. Mötets öppnande}
Vincent förklarar mötet öppnat kl xx.xx.

\section*{§2. Deltagare}
\begin{itemize}
    \item Albin
    \item Axel
    \item Carl
    \item Filip
    \item Vincent
\end{itemize}


\section*{§3. Inledning}
Runda bordet, hur mår alla?
\begin{itemize}
    \item Albin har varit på skidresan, fasttravla hem och sovit första natten i ny lägga
    \item Axel har varit i linkan och är lite stressad.
    \item Carl sprang i lördags och känner av det än. Labbar på i databsaeer
    \item Filip mår ganska bra hög arbetsbelastning i andrakursen 
    \item Vincent mår bra har känt sig lite stressad. Har börjat äta vitaminer 
\end{itemize}


\section*{§4. Arbete sedan senaste mötet}
\begin{itemize}
    \item Hur gick intervjuerna? Key takeaways?\\
            \textit{Kolla trello-kortet}
            \\Intervjuerna \\
            Emelie rundirector, många av volontärerna är äldre och det är ett problem att hitta nya volontärer som gör jobbet korrekt. OS trodde hon var 50/50. Lite oklart var man skulle trycka för att komma till ett event, tyckte att registreringsdelen var onödig. Fler checkpoints behövs på kartan. Hon föreslog att vi skulle skicka notis när volontär är nära dens markör. \\ 
            Nästa intervju tyckte det var svårt med volontärer att få nya. Han ville ha nedskriven info över hur man ska göra. Han ville ha med m.ö.h. \\
            Ryan godsman parkrunner i 10år och volontärat i 7. Även han berättade att det var svårt att förklara för volontärer var man skulle sätta ut markörer. Han ville INTE logga in på något sätt. Han tyckte helst att han inte skulle behöva ta upp appen när han var ute o gick med alla lappar. Han ville hellre att vi skulle ha en informationsdata bas. \\ 
            Angela godsman sa ungefär samma som Ryan men var lite mer förstående.\\ 
            Hon gillade checkpoints, hon tyckte likt emelie att vi borde skicka notiser när man är nära sin markör som ska sättas upp. Hon hade använt appen speciellt när de kör på sin vinterbana \\ \\

            Bildal

            Anna-Carin och Janne \\
            Janne var mer av en löpare än Anna-Carin och båda är volontärer. \\ 
            De ville ha information om kollektivtrafik. De ville ha appen på svenska. Ville att kaffet skulle synas i appen lite konstigt. De var överlag positiva. Under intervjun spann de iväg lite. \\ 
            

            
    \item Har något annat arbete gjorts under helgen?\\
    \item Eventuella utmaningar och hur de har hanterats.\\
    Vill vi skriva om vår tidsplan för att göra ny prototyp. \\ Vi behåller det som det är och reflekterar kring det senare. 
\end{itemize}


\section*{§5. Demo}

\begin{itemize}
    \item Demonstration av den aktuella versionen av appen.\\
    \item Feedback och diskussion.\\
\end{itemize}

\section*{§6. Teknisk diskussion }

\begin{itemize}
    \item Genomgång av de tekniska aspekterna av appen.\\
    
    \item Eventuella tekniska utmaningar och lösningar.\\
    Ta bort register as organizer för de är ganska satta i sten. Ha istället bara en knapp " Volonteer this race" \\
    Bild, notis 
    
\end{itemize}

\section*{§7. Designdiskussion}

\begin{itemize}
    \item Synpunkter på den nuvarande/planerade designen? \\
    \item \textit{Tänka om hela appen?}\\
    
    
\end{itemize}

\section*{§8. Planering av veckans arbete}
\begin{itemize}
    \item Gå igenom Trello-board.\\
    \item Diskussion om kommande arbetsuppgifter och mål.\\
    \item Dela ut uppgifter för veckan.\\
\end{itemize}

\section*{§9. Feedback och frågestund}
\begin{itemize}
    \item Feedback och frågor kring själva arbetet.\\
\end{itemize}
    \section*{§10. Mötets avslutande}
Vincent avslutar mötet kl 
\newpage
\thispagestyle{style2}
\makebox{}\\
\makebox{}\\
\makebox{}\\
\makebox[0.4\linewidth]{\rule{0.4\linewidth}{0.4pt}} \hspace{1cm} \makebox[0.4\linewidth]{\rule{0.4\linewidth}{0.4pt}} \hspace{1cm}\\
\makebox[0.4\linewidth]{Axel Carlstedt} \hspace{1cm}
\makebox[0.4\linewidth]{Vincent Stocks} \hspace{1cm}\\
\makebox[0.4\linewidth]{Mötessekreterare} \hspace{1cm}
\makebox[0.4\linewidth]{Mötesordförande} \hspace{1cm}\\
\makebox{}\\
\makebox{}\\
\makebox{}\\




\end{document}